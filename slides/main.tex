\documentclass[8pt,aspectratio=169]{beamer}
\usetheme{Madrid}
\usecolortheme{default}

\usepackage{graphicx}
\usepackage{amsmath,amssymb}
\usepackage{booktabs}
\usepackage{tikz}

% Custom colors
\definecolor{primary}{RGB}{52, 152, 219}
\definecolor{secondary}{RGB}{46, 204, 113}
\definecolor{accent}{RGB}{231, 76, 60}

\setbeamercolor{structure}{fg=primary}
\setbeamercolor{title}{fg=white,bg=primary}
\setbeamercolor{frametitle}{fg=white,bg=primary}

% Bottom note command
\newcommand{\bottomnote}[1]{%
\vfill\footnotesize\textcolor{gray}{#1}}

\title{Deep Generative Models for Private Credit SPV Analytics}
\subtitle{Cashflow Estimation and Portfolio Loss Distribution}
\author{Digital Finance Research}
\date{\today}

\begin{document}

\begin{frame}
\titlepage
\end{frame}

\begin{frame}[t]{Outline}
\tableofcontents
\end{frame}

\section{Problem Statement}

\begin{frame}[t]{Private Credit SPV Challenge}
\textbf{Special Purpose Vehicles (SPVs) in Private Credit}
\begin{itemize}
\item Securitize portfolios of corporate, consumer, real estate, and trade receivables
\item Need to estimate: loan-level cashflows, portfolio losses, tranche returns
\item Required for: pricing, risk management, regulatory compliance (IFRS 9, Basel)
\end{itemize}
\vspace{0.3cm}
\textbf{Key Modeling Challenges}
\begin{itemize}
\item Heterogeneous loan characteristics across asset classes
\item Macro-economic dependencies in default and prepayment
\item Correlation structure for portfolio-level risk
\end{itemize}
\bottomnote{Traditional Markov models lack flexibility for complex dynamics}
\end{frame}

\begin{frame}[t]{SPV Data Structure}
\begin{center}
\includegraphics[width=0.55\textwidth]{charts/02_data_flow/chart.pdf}
\end{center}
\bottomnote{Originators contribute loan portfolios; SPV tranches absorb losses in priority order}
\end{frame}

\section{Model Architecture}

\begin{frame}[t]{Hierarchical Deep Generative Framework}
\begin{center}
\includegraphics[width=0.58\textwidth]{charts/01_architecture/chart.pdf}
\end{center}
\bottomnote{Four-level hierarchy: Macro $\to$ Cohort $\to$ Loan $\to$ Portfolio}
\end{frame}

\begin{frame}[t]{Level 1: Macro Scenario Generator (VAE)}
\textbf{Conditional Variational Autoencoder for Macro Paths}
\begin{itemize}
\item Generate correlated macro time series: GDP, unemployment, credit spreads
\item Condition on scenario type: baseline, adverse, severely adverse
\item Encoder-decoder with LSTM for temporal dynamics
\end{itemize}
\vspace{0.3cm}
\textbf{Architecture}
\begin{equation*}
z \sim q_\phi(z|x, s), \quad \hat{x} = p_\theta(x|z, s)
\end{equation*}
where $s$ is scenario label, $z$ is latent code, $x$ is macro path.
\bottomnote{Enables joint generation and scenario-conditional stress testing}
\end{frame}

\begin{frame}[t]{Level 2: Transition Transformer}
\textbf{Cohort-Level Transition Matrix Prediction}
\begin{itemize}
\item Transformer encoder processes macro path + cohort features
\item Outputs time-varying transition matrices $P_t$ per cohort
\item Captures systematic risk and macro sensitivity
\end{itemize}
\vspace{0.3cm}
\textbf{Transition Probabilities}
\begin{equation*}
P_t(\text{state}_{t+1} | \text{state}_t, \text{macro}_t, \text{cohort}) = f_\theta(\text{macro}_{1:t}, \text{cohort})
\end{equation*}
\bottomnote{Roll rates, cure rates, prepayment rates vary with economic conditions}
\end{frame}

\begin{frame}[t]{Loan State Transitions}
\begin{center}
\includegraphics[width=0.55\textwidth]{charts/03_model_hierarchy/chart.pdf}
\end{center}
\bottomnote{Seven-state Markov chain with absorbing states for default, prepayment, maturity}
\end{frame}

\begin{frame}[t]{Level 3: Loan Trajectory Model}
\textbf{Autoregressive Generation of Individual Loan Paths}
\begin{itemize}
\item Transformer decoder for state sequence (discrete)
\item Diffusion head for payment amounts (continuous)
\item Hazard rate module for default timing (survival)
\end{itemize}
\vspace{0.3cm}
\textbf{Components}
\begin{equation*}
p(\text{state}_t, \text{payment}_t | \text{history}, \text{loan features}, \text{macro})
\end{equation*}
\bottomnote{Captures heterogeneity within cohorts and idiosyncratic risk}
\end{frame}

\begin{frame}[t]{Level 4: Portfolio Aggregator}
\textbf{Differentiable Waterfall for End-to-End Training}
\begin{itemize}
\item Aggregate loan trajectories to portfolio cashflows
\item Apply waterfall: fees $\to$ senior interest $\to$ mezzanine $\to$ equity
\item Compute tranche-level metrics: IRR, loss rates, coverage ratios
\end{itemize}
\vspace{0.3cm}
\textbf{Loss Distribution}
\begin{equation*}
\text{VaR}_\alpha = F^{-1}_L(\alpha), \quad \text{CVaR}_\alpha = \mathbb{E}[L | L > \text{VaR}_\alpha]
\end{equation*}
\bottomnote{Full distribution enables regulatory capital and pricing calculations}
\end{frame}

\section{Data Schema}

\begin{frame}[t]{Loan Tape Structure}
\textbf{Static Features (At Origination)}
\begin{table}
\footnotesize
\begin{tabular}{lll}
\toprule
\textbf{Category} & \textbf{Features} & \textbf{Examples} \\
\midrule
Loan Terms & Balance, rate, term, amortization & EUR 500k, 6\%, 60mo \\
Underwriting & LTV, DSCR, credit scores & 75\%, 1.3x, 720 \\
Borrower & Industry, geography, type & Manufacturing, DE, SME \\
Collateral & Type, value, lien position & Real estate, EUR 650k \\
\bottomrule
\end{tabular}
\end{table}
\textbf{Time-Varying Features (Monthly Panel)}
\begin{table}
\footnotesize
\begin{tabular}{lll}
\toprule
\textbf{Category} & \textbf{Features} \\
\midrule
Balance & Current balance, scheduled/actual payment \\
Delinquency & Days past due, bucket (30/60/90+), cure flag \\
State & Performing, delinquent, default, prepaid, matured \\
\bottomrule
\end{tabular}
\end{table}
\end{frame}

\begin{frame}[t]{Asset Class Specifications}
\begin{table}
\footnotesize
\begin{tabular}{lcccc}
\toprule
\textbf{Parameter} & \textbf{Corporate} & \textbf{Consumer} & \textbf{Real Estate} & \textbf{Receivables} \\
\midrule
Balance Range & 100k--5M & 1k--100k & 50k--10M & 1k--5M \\
Interest Rate & 4--12\% & 5--20\% & 2--8\% & 3--10\% \\
Term (months) & 12--84 & 6--84 & 60--360 & 1--6 \\
Annual Default & 1--5\% & 2--8\% & 0.5--3\% & 0.5--2\% \\
LGD & 30--60\% & 60--90\% & 15--40\% & 20--50\% \\
\bottomrule
\end{tabular}
\end{table}
\bottomnote{Different risk profiles require asset-class-specific calibration}
\end{frame}

\section{Implementation}

\begin{frame}[t]{Training Strategy}
\textbf{Stage 1: Pre-train Components Separately}
\begin{itemize}
\item Macro VAE on historical macro series
\item Transition Transformer on cohort-level roll data
\item Loan trajectory model on loan-month panel
\end{itemize}
\vspace{0.3cm}
\textbf{Stage 2: End-to-End Fine-Tuning}
\begin{itemize}
\item Joint training with portfolio loss targets
\item Adversarial scenarios for tail calibration
\end{itemize}
\vspace{0.3cm}
\textbf{Stage 3: Calibration}
\begin{itemize}
\item Match historical default rates and loss distribution moments
\item Validate on out-of-sample vintages
\end{itemize}
\end{frame}

\begin{frame}[t]{Outputs and Use Cases}
\begin{table}
\footnotesize
\begin{tabular}{lll}
\toprule
\textbf{Output} & \textbf{Format} & \textbf{Use Case} \\
\midrule
PD Term Structure & Loan $\times$ Month matrix & Pricing, ECL calculation \\
LGD Distribution & Per-loan percentiles & Capital modeling \\
Cashflow Paths & $N_{\text{sim}} \times T$ matrix & DCF valuation \\
Loss Distribution & VaR, CVaR percentiles & Risk limits, capital \\
Tranche Returns & By tranche, by scenario & Investment decisions \\
\bottomrule
\end{tabular}
\end{table}
\vspace{0.3cm}
\textbf{Scenario Capabilities}
\begin{itemize}
\item Baseline: unconditional generation from VAE
\item Deterministic stress: fix macro path, generate loans
\item Probabilistic stress: condition on tail of macro distribution
\item Reverse stress: find macro path producing target loss
\end{itemize}
\end{frame}

\section{Conclusion}

\begin{frame}[t]{Summary}
\textbf{Contribution}
\begin{itemize}
\item Hierarchical deep generative framework for private credit SPV analytics
\item Macro VAE + Transition Transformer + Loan Trajectory + Portfolio Aggregator
\item End-to-end differentiable for joint training and calibration
\end{itemize}
\vspace{0.3cm}
\textbf{Advantages over Traditional Models}
\begin{itemize}
\item Flexible: captures complex macro-credit dependencies
\item Generative: full distribution, not just point estimates
\item Scalable: handles large portfolios with heterogeneous assets
\item Interpretable: hierarchical structure maps to business logic
\end{itemize}
\vspace{0.3cm}
\textbf{Next Steps}
\begin{itemize}
\item Calibration to historical CLO/ABS data
\item Stress testing framework integration
\item Real-time inference for portfolio monitoring
\end{itemize}
\end{frame}

\begin{frame}[t]{References}
\footnotesize
\textbf{Credit Risk}
\begin{itemize}
\item Merton, R.C. (1974). On the pricing of corporate debt
\item CreditMetrics (1997). J.P. Morgan Technical Document
\item CreditRisk+ (1997). Credit Suisse First Boston
\end{itemize}
\textbf{Deep Generative Models}
\begin{itemize}
\item Kingma \& Welling (2014). Auto-Encoding Variational Bayes
\item Ho et al. (2020). Denoising Diffusion Probabilistic Models
\item Vaswani et al. (2017). Attention Is All You Need
\end{itemize}
\textbf{Time Series Generation}
\begin{itemize}
\item Yoon et al. (2019). Time-series Generative Adversarial Networks
\item Tashiro et al. (2021). CSDI: Conditional Score-based Diffusion
\end{itemize}
\end{frame}

\end{document}
