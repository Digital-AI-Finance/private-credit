\documentclass[11pt,a4paper]{article}
\usepackage[margin=2.5cm]{geometry}
\usepackage{amsmath,amssymb,amsthm}
\usepackage{graphicx}
\usepackage{booktabs}
\usepackage{hyperref}
\usepackage{natbib}
\usepackage{algorithm}
\usepackage{algpseudocode}
\usepackage{tikz}
\usetikzlibrary{shapes,arrows,positioning}

\title{Deep Generative Models for Private Credit SPV Analytics:\\
A Hierarchical Framework for Cashflow Estimation and Loss Distribution}

\author{Digital Finance Research Group}

\date{\today}

\begin{document}

\maketitle

\begin{abstract}
We propose a hierarchical deep generative framework for modeling private credit Special Purpose Vehicles (SPVs). The framework integrates four neural network components: (1) a Conditional Variational Autoencoder for macro scenario generation, (2) a Transition Transformer for cohort-level state dynamics, (3) an Autoregressive Loan Trajectory Model with diffusion-based payment generation, and (4) a Differentiable Portfolio Aggregator for waterfall simulation. This architecture enables end-to-end training for joint estimation of loan-level probability of default, portfolio loss distributions, and tranche-level returns under various macro scenarios. We demonstrate the framework on synthetic data spanning corporate loans, consumer credit, real estate, and trade receivables, showing improved calibration and scenario generation compared to traditional Markov chain approaches.
\end{abstract}

\section{Introduction}

Private credit markets have grown substantially in recent years, with assets under management exceeding \$1.5 trillion globally \citep{preqin2024}. A significant portion of this capital flows through Special Purpose Vehicles (SPVs) that securitize portfolios of loans across multiple asset classes. Accurate modeling of these portfolios is essential for:

\begin{itemize}
    \item \textbf{Pricing}: Fair valuation of SPV tranches for primary issuance and secondary trading
    \item \textbf{Risk Management}: Capital allocation, limit setting, and stress testing
    \item \textbf{Regulatory Compliance}: IFRS 9 expected credit loss calculations, Basel capital requirements
    \item \textbf{Investment Decisions}: Risk-adjusted return analysis for tranche selection
\end{itemize}

Traditional approaches rely on Markov chain models for state transitions and Monte Carlo simulation for loss aggregation \citep{creditmetrics1997}. While computationally tractable, these methods face limitations:

\begin{enumerate}
    \item \textbf{Static Transition Matrices}: Conventional models use fixed or discretely-calibrated transition probabilities, failing to capture continuous macro dependencies
    \item \textbf{Limited Heterogeneity}: Loan-level features are often reduced to rating categories, losing granular information
    \item \textbf{Correlation Challenges}: Portfolio-level dependencies are typically modeled through copulas with limited flexibility
    \item \textbf{Point Estimates}: Traditional models focus on expected loss rather than full distributions
\end{enumerate}

We address these limitations through a hierarchical deep generative framework that learns complex dependencies from data while maintaining interpretable structure.

\section{Problem Formulation}

\subsection{Data Structure}

Consider an SPV with $N$ loans observed over $T$ months. For each loan $i$, we observe:

\begin{itemize}
    \item \textbf{Static features} $x_i \in \mathbb{R}^{d_s}$: origination characteristics (balance, rate, LTV, credit score, etc.)
    \item \textbf{Time-varying features} $h_{i,t} \in \mathbb{R}^{d_h}$: monthly performance (balance, payment, delinquency status)
    \item \textbf{State sequence} $s_{i,1:T} \in \mathcal{S}^T$: loan states where $\mathcal{S} = \{\text{performing}, \text{30DPD}, \ldots, \text{default}\}$
\end{itemize}

Additionally, we observe macro-economic conditions $m_t \in \mathbb{R}^{d_m}$ (GDP growth, unemployment, credit spreads) and cohort-level aggregates.

\subsection{Modeling Objectives}

We seek to model the joint distribution:
\begin{equation}
    p(s_{1:N,1:T}, c_{1:N,1:T} | x_{1:N}, m_{1:T})
\end{equation}
where $c_{i,t}$ represents loan $i$'s cashflow at time $t$. From this joint distribution, we derive:

\begin{enumerate}
    \item \textbf{PD Term Structure}: $\text{PD}_i(t) = \Pr(s_{i,\tau} = \text{default for some } \tau \leq t)$
    \item \textbf{Portfolio Loss Distribution}: $\mathcal{L}(L_T)$ where $L_T = \sum_i \mathbb{1}_{s_{i,T}=\text{default}} \cdot \text{LGD}_i \cdot \text{EAD}_i$
    \item \textbf{Tranche Returns}: $R_k = f_k(\{c_{i,t}\}_{i,t})$ through waterfall mechanics
\end{enumerate}

\section{Methodology}

\subsection{Hierarchical Architecture}

Our framework decomposes the joint distribution hierarchically:

\begin{equation}
    p(s, c | x, m) = \underbrace{p(m)}_{\text{Macro VAE}} \cdot \underbrace{p(P | m, z)}_{\text{Transition Transformer}} \cdot \underbrace{p(s, c | x, P, m)}_{\text{Loan Trajectory}} \cdot \underbrace{p(L, R | s, c)}_{\text{Aggregator}}
\end{equation}

\subsection{Level 1: Macro Scenario Generator (VAE)}

We model macro paths using a Conditional Variational Autoencoder \citep{kingma2014}:

\begin{align}
    \text{Encoder:} \quad & z \sim q_\phi(z | m_{1:T}, s_{\text{scenario}}) = \mathcal{N}(\mu_\phi, \Sigma_\phi) \\
    \text{Decoder:} \quad & \hat{m}_{1:T} \sim p_\theta(m | z, s_{\text{scenario}})
\end{align}

where $s_{\text{scenario}} \in \{\text{baseline}, \text{adverse}, \text{severely\_adverse}\}$ is a scenario label enabling conditional generation.

The encoder uses a bidirectional LSTM to process the macro sequence, while the decoder autoregressively generates future paths. Training minimizes:

\begin{equation}
    \mathcal{L}_{\text{VAE}} = \mathbb{E}_{q_\phi}[\|m - \hat{m}\|^2] + \beta \cdot D_{\text{KL}}(q_\phi(z|m) \| p(z))
\end{equation}

\subsection{Level 2: Transition Transformer}

Given macro conditions and cohort features, we predict time-varying transition matrices using a Transformer encoder \citep{vaswani2017}:

\begin{equation}
    P_t = \text{TransitionTransformer}(m_{1:t}, z_{\text{cohort}})
\end{equation}

where $P_t \in \mathbb{R}^{|\mathcal{S}| \times |\mathcal{S}|}$ is a row-stochastic transition matrix.

The architecture uses:
\begin{itemize}
    \item Multi-head self-attention over temporal dimension
    \item Cross-attention between macro sequence and cohort embedding
    \item Separate output heads for each source state, with softmax normalization
\end{itemize}

\subsection{Level 3: Loan Trajectory Model}

For individual loan paths, we combine:

\begin{enumerate}
    \item \textbf{Discrete State Head}: Autoregressive prediction of next state
    \begin{equation}
        p(s_{i,t+1} | s_{i,1:t}, x_i, m_{1:t}) = \text{Softmax}(f_\theta^{\text{state}}(h_t))
    \end{equation}

    \item \textbf{Continuous Payment Head}: Diffusion-based generation \citep{ho2020}
    \begin{equation}
        c_{i,t} \sim p_\theta(c | h_t) \quad \text{via reverse diffusion}
    \end{equation}

    \item \textbf{Hazard Rate Module}: Survival analysis component
    \begin{equation}
        \lambda_i(t) = \sigma(f_\theta^{\text{hazard}}(h_t))
    \end{equation}
\end{enumerate}

The Transformer decoder processes loan embeddings, previous states, and macro context.

\subsection{Level 4: Portfolio Aggregator}

Loan-level trajectories are aggregated to portfolio and tranche level:

\begin{equation}
    L_t = \sum_{i=1}^N \mathbb{1}_{s_{i,t}=\text{default}} \cdot \text{LGD}_i \cdot B_{i,t}
\end{equation}

where $B_{i,t}$ is the balance at default.

Tranche cashflows are computed via waterfall:
\begin{equation}
    R_k = \sum_t \text{Waterfall}_k(\{c_{i,t}\}_i, \{L_{i,t}\}_i)
\end{equation}

For end-to-end training, we implement a differentiable waterfall using soft assignments.

\section{Training Procedure}

\subsection{Stage 1: Pre-training}

Each component is pre-trained independently:
\begin{enumerate}
    \item Macro VAE on historical macro time series
    \item Transition Transformer on cohort-level roll rates
    \item Loan Trajectory Model on loan-month panel data
\end{enumerate}

\subsection{Stage 2: End-to-End Fine-tuning}

Components are jointly fine-tuned with portfolio-level objectives:
\begin{equation}
    \mathcal{L}_{\text{E2E}} = \mathcal{L}_{\text{state}} + \lambda_1 \mathcal{L}_{\text{payment}} + \lambda_2 \mathcal{L}_{\text{loss}} + \lambda_3 \mathcal{L}_{\text{tail}}
\end{equation}

where $\mathcal{L}_{\text{tail}}$ emphasizes calibration in distribution tails via quantile regression.

\subsection{Stage 3: Calibration}

Final calibration matches:
\begin{itemize}
    \item Historical default rates by cohort and asset class
    \item Loss distribution moments and quantiles
    \item Out-of-sample vintage performance
\end{itemize}

\section{Experimental Design}

\subsection{Data}

We generate synthetic data with realistic characteristics:
\begin{itemize}
    \item $N = 10,000$ loans across 4 asset classes
    \item $T = 60$ months observation window
    \item 24 origination cohorts (vintages)
    \item Macro scenarios: baseline, adverse, severely adverse, stagflation
\end{itemize}

\subsection{Baselines}

We compare against:
\begin{enumerate}
    \item Multi-state Markov chain with fixed transition matrices
    \item Markov chain with macro-adjusted transitions (industry standard)
    \item CreditMetrics-style simulation with copula dependence
\end{enumerate}

\subsection{Evaluation Metrics}

\begin{table}[h]
\centering
\caption{Evaluation Metrics}
\begin{tabular}{lll}
\toprule
\textbf{Metric} & \textbf{Description} & \textbf{Target} \\
\midrule
Default Rate Calibration & $|\hat{\text{DR}} - \text{DR}| / \text{DR}$ & $< 10\%$ \\
KS Test (Loss Distribution) & Kolmogorov-Smirnov vs historical & $p > 0.05$ \\
Tail Calibration & VaR 99\% accuracy & $\pm 15\%$ \\
Scenario Sensitivity & $\Delta \text{Loss} / \Delta \text{Macro}$ & Economically plausible \\
\bottomrule
\end{tabular}
\end{table}

\section{Expected Contributions}

\begin{enumerate}
    \item \textbf{Methodological}: First application of hierarchical deep generative models to private credit SPV analytics
    \item \textbf{Practical}: End-to-end framework for pricing, risk management, and regulatory compliance
    \item \textbf{Empirical}: Comprehensive comparison with industry-standard approaches on realistic scenarios
\end{enumerate}

\section{Timeline}

\begin{enumerate}
    \item \textbf{Phase 1}: Data schema and synthetic data generation
    \item \textbf{Phase 2}: Baseline model implementation and benchmarking
    \item \textbf{Phase 3}: Deep generative component development
    \item \textbf{Phase 4}: Integration, calibration, and validation
    \item \textbf{Phase 5}: Documentation and dissemination
\end{enumerate}

\section{Conclusion}

We propose a hierarchical deep generative framework that addresses key limitations of traditional credit risk models. By learning complex macro-credit dependencies, capturing loan-level heterogeneity, and enabling full distribution estimation, our approach provides a flexible and powerful tool for private credit SPV analytics.

\bibliographystyle{apalike}
\begin{thebibliography}{10}

\bibitem[CreditMetrics, 1997]{creditmetrics1997}
J.P. Morgan (1997).
\newblock CreditMetrics Technical Document.

\bibitem[Ho et al., 2020]{ho2020}
Ho, J., Jain, A., \& Abbeel, P. (2020).
\newblock Denoising Diffusion Probabilistic Models.
\newblock {\em NeurIPS}.

\bibitem[Kingma \& Welling, 2014]{kingma2014}
Kingma, D. P., \& Welling, M. (2014).
\newblock Auto-Encoding Variational Bayes.
\newblock {\em ICLR}.

\bibitem[Preqin, 2024]{preqin2024}
Preqin (2024).
\newblock Global Private Debt Report.

\bibitem[Vaswani et al., 2017]{vaswani2017}
Vaswani, A., et al. (2017).
\newblock Attention Is All You Need.
\newblock {\em NeurIPS}.

\end{thebibliography}

\end{document}
